% !TEX program = lualatex
% !TEX root = ./tlinstall.tex

\documentclass[ structure      = article
              , maketitlestyle = standard
              , secstyle       = center
              , secfont        = roman
              , liststyle      = aligned
              , quotesize      = normalsize
              , headerstyle    = center
              , footnotestyle  = superscript
              ]{suftesi}

% gestione font
\usepackage[no-math]{fontspec}
\setmainfont{Linux Libertine O}
\setmonofont{Ubuntu Mono}[Scale=MatchUppercase]
\usepackage{anyfontsize}

% gestione lingue
\usepackage{polyglossia}
\setmainlanguage{italian}

% virgolette
\usepackage[autostyle,italian=quotes]{csquotes}
\newcommand\q\enquote

% riferimenti incrociati
\usepackage{hyperref}
\hypersetup{ breaklinks
           , hidelinks
           , linktocpage
           }

% bibliografia
\usepackage[backend=biber]{biblatex}
\addbibresource{mybib.bib}
\nocite{*}

% immagini
\usepackage{graphicx}
%\graphicspath{{figure}}

% codici
\usepackage{listings}
\lstset{ basicstyle      = \ttfamily
       , escapeinside    = {?!}{!?}
       , escapebegin     = {\normalfont\em$\langle$\kern-1pt}
       , escapeend       = {$\rangle$}
       , backgroundcolor = \color{gray!10!white}
       }

% abstract
\usepackage{abstract}

% comandi personali
\newcommand\texlive{\TeX{}{\em live}}
\newcommand\gnulinux{{\sf GNU/Linux}}

% tasti tastiera
\usepackage{menukeys}
\newcommand\invio{\keys\return}
