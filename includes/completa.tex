% !TEX program = lualatex
% !TEX spellcheck = it_IT
% !TEX root = ../tlinstall.tex

\section{Installazione completa}

Avvisiamo da subito, però, che questa via comporta uno svantaggio non da poco. Allo stato delle cose questa distribuzione ingombra almeno 3GB sul disco, e chi chiede una installazione più snella non ha tutti i torti: c'è una mole enorme di pacchetti che un utente medio probabilmente non userà mai.

\begin{enumerate}

\item Bisogna scaricare il file \lstinline£texlive.iso£. Andare su
\begin{center}
\url{http://www.tug.org/texlive/acquire-iso.html}
\end{center}
e poi cliccare su {\sf Download from a nearby CTAN mirror}. Su questo primo passo c'è da dire una cosa: occorre armarsi di mooolta pazienza\dots{} Questo è forse l'ostacolo maggiore che si può incontrare.

\item Creiamo la cartella \lstinline£iso£ all'interno della cartella \lstinline£media£
\begin{lstlisting}
sudo mkdir /media/iso
\end{lstlisting}
e montiamoci \lstinline£texlive.iso£
\begin{lstlisting}
sudo mount -o loop ?!dove si trova!?/texlive.iso /media/iso
\end{lstlisting}
Di solito quello che si scarica col browser va a finire in \lstinline£~/Scaricati£.

\item Possiamo iniziare l'installazione
\begin{lstlisting}
sudo /media/iso/install-tl
\end{lstlisting}
Durante questo processo non sono richiesti particolari interventi dall'utente: quando viene chiesto dal terminale quale tipo di installazione effettuare, scegliere quella completa digitando {\sf I} (la lettera \q{i} maiuscola) e premendo il tasto \invio{}.\footnote{Questo passaggio per essere portato al termine con successo richiede un po' meno tempo: dai 15 ai 20 minuti.} Potrebbe essere utile sapere che tutto quello che viene installato di default va a finire in \lstinline£/usr/local/texlive£.

\item Finita l'installazione \lstinline£texlive.iso£ non ci serve più, perciò smontiamolo
\begin{lstlisting}
sudo umount /media/iso
\end{lstlisting}
e possiamo eliminare anche la cartella \lstinline£/media/iso£
\begin{lstlisting}
sudo rm -rf /media/iso
\end{lstlisting}

\item Impartiamo il seguente comando
\begin{lstlisting}
xdg-open ~/.bash_aliases
\end{lstlisting}
che apre una pagina di file di testo.\footnote{Questo file può darsi che non esista (anzi è probabile che non esista) e quindi bisogna crearselo: si può fare col comando \lstinline£touch ~/.bash_aliases£.} Copiamo alla fine del file che viene aperto le seguenti righe:
\begin{lstlisting}
texlive="/usr/local/texlive/?!anno!?"
PATH=$texlive/bin/x86_64-linux:$PATH
export PATH
MANPATH=$texlive/texmf-dist/doc/man:$MANPATH
export MANPATH
INFOPATH=$texlive/texmf-dist/doc/info:$INFOPATH
export INFOPATH

alias tlmgr='sudo env PATH=$PATH tlmgr'
\end{lstlisting}
Salviamo ed chiudiamo la finestra. Facciamo assimilare le novità introdotte:
\begin{lstlisting}
. ~/.bash_aliases
\end{lstlisting}

\end{enumerate}
