% !TEX program = lualatex
% !TEX spellcheck = it_IT
% !TEX root = ../tlinstall.tex

\section{Installazione minimale}

In questa sezione vogliamo fornire un'alternativa forse più praticabile. Installeremo una \texlive{} molto minimale ed anche se questa base è troppo povera, può essere tranquillamente arricchita installando pacchetti. Avvertiamo che se si intende seguire questa strada, bisogna essere connessi ad internet durante tutta la procedura.

Scarichiamo \lstinline£install-tl-unx.tar.gz£ dal sito internet
\begin{center}
\url{https://tug.org/texlive/acquire-netinstall.html},
\end{center}
decomprimiamola e apriamo il terminale all'interno della cartella estratta. A questo punto iniziamo l'installazione:

\begin{lstlisting}
sudo ./install-tl --scheme minimal
\end{lstlisting}

Osserviamo che questo passo richiede i privilegi di amministrazione. L'unico intervento da parte dell'utente è quello di premere i pulsanti \lstinline£I£ e \lstinline£Invio£ per confermare.

Una volta che \lstinline£install-tl£ ha terminato, fare quanto indicato nel punto (5) della sezione precedente.

Possiamo così ora iniziare il lavoro di post-installazione. Nella nostra distribuzione non sono presenti ad esempio \lstinline£pdflatex£, \lstinline£lualatex£ e \lstinline£xelatex£, che invece sono molto usati. Li si installa con

\begin{lstlisting}
tlmgr install latex
\end{lstlisting}

Una programma molto comodo da avere è \lstinline£texliveonfly£: scritto in \lstinline£Python£, durante la creazione dei nostri documenti installa i pacchetti che mancano. Il suo uso è molto semplice:
\begin{lstlisting}
texliveonfly -c ?!compilatore!? ?!file da compilare!?
\end{lstlisting}
dove {\em compilatore} può essere ad esempio \lstinline£pdflatex£, \lstinline£lualatex£ oppure \lstinline£xelatex£. Se lo si vuole installare,

\begin{lstlisting}
tlmgr install texliveonfly
\end{lstlisting}

Avrete capito, insomma, che i nuovi pacchetti si installano proprio così
\begin{lstlisting}
tlmgr install ?!nome pacchetto!?
\end{lstlisting}
