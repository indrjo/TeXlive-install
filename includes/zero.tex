% !TEX program = lualatex
% !TEX spellcheck = it_IT
% !TEX root = ../tlinstall.tex

\section*{Abstract}

In queste pagine vengono spiegate due modalità di installazione, una minimale ed un'altra completa. Ognuna delle due ha i propri pregi ed i propri difetti. Leggere le rispettive sezioni per decidere cos'è meglio.

%\section*{Prerequisiti (per principianti)}
%
%\begin{figure}
%\centering
%\includegraphics[width=.7\textwidth]{terminale}
%\caption{Un esempio di finestra di terminale. (Screenshot da {\sf elementaryOS}).}
%\label{fig:terminale}
%\end{figure}
%
%Si può essere nuovi al mondo \gnulinux{} o impacciati col computer in generale, ma questa cosa è davvero alla portata di tutti: avviare il terminale (un programma che si presenta come in figura~\ref{fig:terminale}) e dare delle istruzioni attraverso la tastiera. Questo programma non è molto elaborato (ma è molto potente, non ti far ingannare): all'avvio si presenta solo la riga
%\begin{lstlisting}
%?!...!?@?!...!?:?!...!?$
%\end{lstlisting}
%seguita da un rettangolino lampeggiante (oppure no, dipende\dots{}): significa che il terminale è in ascolto ed è disponibile a ricevere istruzioni. Il tasto \invio{} fa eseguire i comandi che abbiamo scritto.
%
%Per quello che ci serve in seguito, bisogna inoltre procurarsi i permessi di amministrazione, perché in seguito daremo molti comandi del tipo
%\begin{lstlisting}
%sudo ?!comando!?
%\end{lstlisting}
%Ogni volta che viene digitato una frase di questo tipo sul terminale e premuto il tasto \invio{}, compare come riga successiva la richiesta di inserire la password di amministrazione
%\begin{lstlisting}
%[sudo] password di ?!amministratore!?:
%\end{lstlisting}
