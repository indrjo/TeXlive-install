% !TEX program = lualatex
% !TEX spellcheck = it_IT
% !TEX root = ../tlinstall.tex

\section{Aggiornare i pacchetti}

%\begin{figure}
%\centering
%\includegraphics[width=.7\textwidth]{tlmgr}
%\caption{\tt tlmgr gui}
%\label{fig:gui}
%\end{figure}

Per aggiornare \lstinline£tlmgr£ stesso basta dare
\begin{lstlisting}
tlmgr update --self
\end{lstlisting}
Per sapere quali pacchetti hanno bisogno di essere aggiornati fare eseguire questo comando
\begin{lstlisting}
tlmgr update --list
\end{lstlisting}
Per aggiornare tutti i pacchetti aggiornabili
\begin{lstlisting}
tlmgr update --all
\end{lstlisting}
Il consiglio è, finita la procedura di installazione, di aggiornare {\sf tlmgr} stesso e di aggiornare, se possibile, tutti i pacchetti aggiornabili: questo perché nella \lstinline£texlive.iso£ tutto è congelato allo stesso e identico stato del rilascio, col rischio di trovarsi materiale non aggiornato. Se si vuole aggiornare un singolo pacchetto, si può usare
\begin{lstlisting}
tlmgr update ?!nome pacchetto!?
\end{lstlisting}

%\begin{nota}
%Se il terminale fa ancora paura, si può optare per questa alternativa
%\begin{lstlisting}
%tlmgr gui
%\end{lstlisting}
%che apre la finestra in figura~\ref{fig:gui}. Da lì si possono fare tutte le azioni appena viste e anche altre. Serve che ci sia il modulo {\sf Tk}, però: per installarlo da terminale
%\begin{lstlisting}
%cpan -f Tk
%\end{lstlisting}
%\end{nota}
